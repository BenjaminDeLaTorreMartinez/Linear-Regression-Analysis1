% Options for packages loaded elsewhere
\PassOptionsToPackage{unicode}{hyperref}
\PassOptionsToPackage{hyphens}{url}
\documentclass[
  11pt,
]{article}
\usepackage{xcolor}
\usepackage[margin=1in, paperwidth=8.5in, paperheight=11in]{geometry}
\usepackage{amsmath,amssymb}
\setcounter{secnumdepth}{-\maxdimen} % remove section numbering
\usepackage{iftex}
\ifPDFTeX
  \usepackage[T1]{fontenc}
  \usepackage[utf8]{inputenc}
  \usepackage{textcomp} % provide euro and other symbols
\else % if luatex or xetex
  \usepackage{unicode-math} % this also loads fontspec
  \defaultfontfeatures{Scale=MatchLowercase}
  \defaultfontfeatures[\rmfamily]{Ligatures=TeX,Scale=1}
\fi
\usepackage{lmodern}
\ifPDFTeX\else
  % xetex/luatex font selection
\fi
% Use upquote if available, for straight quotes in verbatim environments
\IfFileExists{upquote.sty}{\usepackage{upquote}}{}
\IfFileExists{microtype.sty}{% use microtype if available
  \usepackage[]{microtype}
  \UseMicrotypeSet[protrusion]{basicmath} % disable protrusion for tt fonts
}{}
\makeatletter
\@ifundefined{KOMAClassName}{% if non-KOMA class
  \IfFileExists{parskip.sty}{%
    \usepackage{parskip}
  }{% else
    \setlength{\parindent}{0pt}
    \setlength{\parskip}{6pt plus 2pt minus 1pt}}
}{% if KOMA class
  \KOMAoptions{parskip=half}}
\makeatother
\usepackage{color}
\usepackage{fancyvrb}
\newcommand{\VerbBar}{|}
\newcommand{\VERB}{\Verb[commandchars=\\\{\}]}
\DefineVerbatimEnvironment{Highlighting}{Verbatim}{commandchars=\\\{\}}
% Add ',fontsize=\small' for more characters per line
\usepackage{framed}
\definecolor{shadecolor}{RGB}{248,248,248}
\newenvironment{Shaded}{\begin{snugshade}}{\end{snugshade}}
\newcommand{\AlertTok}[1]{\textcolor[rgb]{0.94,0.16,0.16}{#1}}
\newcommand{\AnnotationTok}[1]{\textcolor[rgb]{0.56,0.35,0.01}{\textbf{\textit{#1}}}}
\newcommand{\AttributeTok}[1]{\textcolor[rgb]{0.13,0.29,0.53}{#1}}
\newcommand{\BaseNTok}[1]{\textcolor[rgb]{0.00,0.00,0.81}{#1}}
\newcommand{\BuiltInTok}[1]{#1}
\newcommand{\CharTok}[1]{\textcolor[rgb]{0.31,0.60,0.02}{#1}}
\newcommand{\CommentTok}[1]{\textcolor[rgb]{0.56,0.35,0.01}{\textit{#1}}}
\newcommand{\CommentVarTok}[1]{\textcolor[rgb]{0.56,0.35,0.01}{\textbf{\textit{#1}}}}
\newcommand{\ConstantTok}[1]{\textcolor[rgb]{0.56,0.35,0.01}{#1}}
\newcommand{\ControlFlowTok}[1]{\textcolor[rgb]{0.13,0.29,0.53}{\textbf{#1}}}
\newcommand{\DataTypeTok}[1]{\textcolor[rgb]{0.13,0.29,0.53}{#1}}
\newcommand{\DecValTok}[1]{\textcolor[rgb]{0.00,0.00,0.81}{#1}}
\newcommand{\DocumentationTok}[1]{\textcolor[rgb]{0.56,0.35,0.01}{\textbf{\textit{#1}}}}
\newcommand{\ErrorTok}[1]{\textcolor[rgb]{0.64,0.00,0.00}{\textbf{#1}}}
\newcommand{\ExtensionTok}[1]{#1}
\newcommand{\FloatTok}[1]{\textcolor[rgb]{0.00,0.00,0.81}{#1}}
\newcommand{\FunctionTok}[1]{\textcolor[rgb]{0.13,0.29,0.53}{\textbf{#1}}}
\newcommand{\ImportTok}[1]{#1}
\newcommand{\InformationTok}[1]{\textcolor[rgb]{0.56,0.35,0.01}{\textbf{\textit{#1}}}}
\newcommand{\KeywordTok}[1]{\textcolor[rgb]{0.13,0.29,0.53}{\textbf{#1}}}
\newcommand{\NormalTok}[1]{#1}
\newcommand{\OperatorTok}[1]{\textcolor[rgb]{0.81,0.36,0.00}{\textbf{#1}}}
\newcommand{\OtherTok}[1]{\textcolor[rgb]{0.56,0.35,0.01}{#1}}
\newcommand{\PreprocessorTok}[1]{\textcolor[rgb]{0.56,0.35,0.01}{\textit{#1}}}
\newcommand{\RegionMarkerTok}[1]{#1}
\newcommand{\SpecialCharTok}[1]{\textcolor[rgb]{0.81,0.36,0.00}{\textbf{#1}}}
\newcommand{\SpecialStringTok}[1]{\textcolor[rgb]{0.31,0.60,0.02}{#1}}
\newcommand{\StringTok}[1]{\textcolor[rgb]{0.31,0.60,0.02}{#1}}
\newcommand{\VariableTok}[1]{\textcolor[rgb]{0.00,0.00,0.00}{#1}}
\newcommand{\VerbatimStringTok}[1]{\textcolor[rgb]{0.31,0.60,0.02}{#1}}
\newcommand{\WarningTok}[1]{\textcolor[rgb]{0.56,0.35,0.01}{\textbf{\textit{#1}}}}
\usepackage{longtable,booktabs,array}
\usepackage{calc} % for calculating minipage widths
% Correct order of tables after \paragraph or \subparagraph
\usepackage{etoolbox}
\makeatletter
\patchcmd\longtable{\par}{\if@noskipsec\mbox{}\fi\par}{}{}
\makeatother
% Allow footnotes in longtable head/foot
\IfFileExists{footnotehyper.sty}{\usepackage{footnotehyper}}{\usepackage{footnote}}
\makesavenoteenv{longtable}
\usepackage{graphicx}
\makeatletter
\newsavebox\pandoc@box
\newcommand*\pandocbounded[1]{% scales image to fit in text height/width
  \sbox\pandoc@box{#1}%
  \Gscale@div\@tempa{\textheight}{\dimexpr\ht\pandoc@box+\dp\pandoc@box\relax}%
  \Gscale@div\@tempb{\linewidth}{\wd\pandoc@box}%
  \ifdim\@tempb\p@<\@tempa\p@\let\@tempa\@tempb\fi% select the smaller of both
  \ifdim\@tempa\p@<\p@\scalebox{\@tempa}{\usebox\pandoc@box}%
  \else\usebox{\pandoc@box}%
  \fi%
}
% Set default figure placement to htbp
\def\fps@figure{htbp}
\makeatother
\setlength{\emergencystretch}{3em} % prevent overfull lines
\providecommand{\tightlist}{%
  \setlength{\itemsep}{0pt}\setlength{\parskip}{0pt}}
\usepackage{float}     % allows [H] figure placement
\usepackage{booktabs}
\usepackage{longtable}
\usepackage{array}
\usepackage{multirow}
\usepackage{wrapfig}
\usepackage{float}
\usepackage{colortbl}
\usepackage{pdflscape}
\usepackage{tabu}
\usepackage{threeparttable}
\usepackage{threeparttablex}
\usepackage[normalem]{ulem}
\usepackage{makecell}
\usepackage{xcolor}
\usepackage{bookmark}
\IfFileExists{xurl.sty}{\usepackage{xurl}}{} % add URL line breaks if available
\urlstyle{same}
\hypersetup{
  pdftitle={A Statistical Approach to Predicting Emphysema Percentage},
  pdfauthor={Benjamin De La Torre Martinez},
  hidelinks,
  pdfcreator={LaTeX via pandoc}}

\title{A Statistical Approach to Predicting Emphysema Percentage}
\usepackage{etoolbox}
\makeatletter
\providecommand{\subtitle}[1]{% add subtitle to \maketitle
  \apptocmd{\@title}{\par {\large #1 \par}}{}{}
}
\makeatother
\subtitle{Connecting Performance with Contract Value}
\author{Benjamin De La Torre Martinez}
\date{}

\begin{document}
\maketitle

\begin{center}
STAT 610\\
Fall 2025\\
2020-12-08
\end{center}

\subsection{Executive Summary}\label{executive-summary}

Emphysema is a progressive lung disease that damages alveoli
(microscopic air sacs fundamental to the respiratory system). While the
precise figure is uncertain, an estimated 1.6\% of adults had a
diagnosis of emphysema. This study thus attempts to identify the set of
combined factors that are most strongly associated with the percentage
of emphysema an indiviual has in their lungs. After completing analysis
through linear regression we can observe that the variables that are
most correlation with percentage of emphysema are those related to the
health of one's respiratory system. More specifically prior diagnoses of
copd and emphysema. Additionally, the health related metrics included in
our model, such as smoking status, average cigarettes smoked per day,
smoking duration, functional residual capacity, inspiratory and
expiratory mean attenuation, gas trapping percentage, the FEV1/FVC
ratio, and FVC, all contribute meaningful information about a patient's
respiratory condition. These variables capture both physiological
function and exposure-related effects, helping to further explain
variation in emphysema percentage beyond prior diagnoses alone. Or in
other words, the worse an individual's respiratory health is, the higher
the percentage of emphysema tends to be.

\newpage

\subsection{Introduction}\label{introduction}

Most sources identify smoking and long-term exposure to second-hand
smoke as the leading causes of emphysema. Rather than proving causation,
this study focuses on a related question: Which variables show a
significant correlation with emphysema percentage? To investigate this,
we examine several indicators of general health (BMI, heart rate, blood
pressure) and respiratory health. Our hypothesis is that emphysema
severity is associated with negative respiratory conditions (such as
COPD and prior emphysema diagnoses), smoking quantity and duration, and
lung-function metrics, though the specific contributions of each remain
uncertain. We apply multivariate linear regression to predict continuous
emphysema percentages (0--100\%) using both continuous and categorical
variables. This method is appropriate for our data structure, and
diagnostic checks will confirm whether its assumptions are satisfied.
While other analytical approaches exist, linear regression is chosen for
its interpretability and its ability to highlight clear relationships
between predictors and emphysema percentage. Although regression assumes
mainly additive effects---a limitation given the complexity of
respiratory interactions---we prioritize interpretability over
predictive accuracy, making regression a justified choice for this
analysis.

\subsection{Exploratory Data Analysis}\label{exploratory-data-analysis}

Our study begins by examining our data. By doing so, we get to
understand the structure, distribution, range and patterns of out data.
Ensuring that our analysis is consistent and account for the variability
of our data.

\subsubsection{A. Summary of our Data}\label{a.-summary-of-our-data}

Our dataset contains measurements of 5747 different individuals
containing 35 predictors. One disclaimer about our data is that our
study seems to be highly biased as subjects are either Caucasian or
African American.

\begin{longtable}[]{@{}ll@{}}
\caption{Data summary}\tabularnewline
\toprule\noalign{}
\endfirsthead
\endhead
\bottomrule\noalign{}
\endlastfoot
Name & df\_copd \\
Number of rows & 5747 \\
Number of columns & 35 \\
\_\_\_\_\_\_\_\_\_\_\_\_\_\_\_\_\_\_\_\_\_\_\_ & \\
Column type frequency: & \\
character & 2 \\
factor & 11 \\
numeric & 22 \\
\_\_\_\_\_\_\_\_\_\_\_\_\_\_\_\_\_\_\_\_\_\_\_\_ & \\
Group variables & None \\
\end{longtable}

Based on the distributional profiles of all variables, no anomalies are
evident. The variables exhibit mild right skewness, mild left skewness,
or near-normal distributions. With the exception of visit\_age which
roughly resembles a uniform distribution.\\
\textbf{Disclaimer: } Not all variables are shown, only a subset.

\begin{longtable}[]{@{}
  >{\centering\arraybackslash}p{(\linewidth - 14\tabcolsep) * \real{0.2317}}
  >{\centering\arraybackslash}p{(\linewidth - 14\tabcolsep) * \real{0.1098}}
  >{\centering\arraybackslash}p{(\linewidth - 14\tabcolsep) * \real{0.1220}}
  >{\centering\arraybackslash}p{(\linewidth - 14\tabcolsep) * \real{0.0976}}
  >{\centering\arraybackslash}p{(\linewidth - 14\tabcolsep) * \real{0.0976}}
  >{\centering\arraybackslash}p{(\linewidth - 14\tabcolsep) * \real{0.1220}}
  >{\centering\arraybackslash}p{(\linewidth - 14\tabcolsep) * \real{0.1098}}
  >{\centering\arraybackslash}p{(\linewidth - 14\tabcolsep) * \real{0.1098}}@{}}
\toprule\noalign{}
\begin{minipage}[b]{\linewidth}\centering
variable
\end{minipage} & \begin{minipage}[b]{\linewidth}\centering
unique
\end{minipage} & \begin{minipage}[b]{\linewidth}\centering
missing
\end{minipage} & \begin{minipage}[b]{\linewidth}\centering
mean
\end{minipage} & \begin{minipage}[b]{\linewidth}\centering
sd
\end{minipage} & \begin{minipage}[b]{\linewidth}\centering
min
\end{minipage} & \begin{minipage}[b]{\linewidth}\centering
median
\end{minipage} & \begin{minipage}[b]{\linewidth}\centering
max
\end{minipage} \\
\midrule\noalign{}
\endhead
\bottomrule\noalign{}
\endlastfoot
visit\_age & 392 & 0 & 59.75 & 8.688 & 39 & 59.5 & 85 \\
bmi & 2052 & 0 & 29.08 & 6.141 & 12.67 & 28.2 & 64.1 \\
CigPerDaySmokAvg & 47 & 0 & 23.78 & 11.42 & 0 & 20 & 99 \\
SmokStartAge & 45 & 0 & 16.74 & 4.821 & 0 & 16 & 50 \\
pct\_gastrapping & 4089 & 1650 & 19.81 & 17.72 & 0.04412 & 13.63 &
81.27 \\
\end{longtable}

\subsubsection{B. Visualizing Predcitors and
Response}\label{b.-visualizing-predcitors-and-response}

\subsubsection{0. Handling Missingness}\label{handling-missingness}

Before preforming any meaningful statistical analysis, it is vital that
we first analyse our missing observations and determine their nature.
For a more detailed explanation please consult the appendix. In essence
because there is a great proportion of missing variables (specifically
the response variable) it has been decided that our analysis will be
conducted on two data set: one in which we impute missing the missing
dependent variable (\textbf{complete}) and one where observations with
missing dependent variable with not be taken into account
(\textbf{imputed}).

\subsubsection{i. Transformation of the explanatory
variables}\label{i.-transformation-of-the-explanatory-variables}

Based on the scatter plots of predictors vs pct\_emphysema most
relationships seem to be either non existent, linear, quadratic or
\(e^{-x}\). At this stage functional relationships cannot be determine
but plots give us an idea on what they could be (functional
relationships can changed as more variables are included in the
model).\\
\textbf{Note:} The following plots only represent a subset of variables
from the data set.

\pandocbounded{\includegraphics[keepaspectratio]{RegressionFinalProject_files/figure-latex/explanatoryTransformationPlots2-1.pdf}}

\subsubsection{ii. Transformation of the
response}\label{ii.-transformation-of-the-response}

The output below consists of three plots: A histogram of the
distribution of the response variable (percentage of emphysema), a
Box-Cox Power Transformation Plot and a histogram of the distribution of
the response variable after applying the optimal box cox transformation.
For both \textbf{complete} and \textbf{imputed} data sets the optimal
power transformation is \(\dfrac{1}{5} = 0.2\)

\pandocbounded{\includegraphics[keepaspectratio]{RegressionFinalProject_files/figure-latex/boxcoxTransformation-1.pdf}}
\pandocbounded{\includegraphics[keepaspectratio]{RegressionFinalProject_files/figure-latex/boxcoxTransformation-2.pdf}}

\subsection{Variable Transformation and
Selection}\label{variable-transformation-and-selection}

\subsubsection{A. Transformation of the explanatory variables: Part
2}\label{a.-transformation-of-the-explanatory-variables-part-2}

In this variable transformation pipeline, we preserve the raw original
data for all variables while adding selected transformations for
specific lung-function measures. These include square-root and log
transforms for lung capacity variables, quadratic terms for mean
attenuation metrics, and both logarithmic and shifted
negative-exponential transforms for FEV1 in phase 2, enabling flexible
modeling while maintaining the original scale for baseline
comparisons.\\
Our best subsets code enforces a mutual exclusivity rule: for variables
with multiple transformations (e.g., raw, sqrt, ln), only one version
can be chosen per model. This ensures we avoid multicollinearity and
select the most useful transformation without redundancy.

\subsubsection{B. Best subset selection +
LASSO}\label{b.-best-subset-selection-lasso}

Taking into account that our data only contains around 30 informative
predictors for regression, utilizing the power of best subset selection
is actually feasible. Our results showed that out of the 30 predictors,
only 19 of them where meaningful. Yet, at this stage it is still to
early to report those as we first need to check whether there exist
multicolinearity in our proposed model. So, we ran LASSO on both sets of
data \textbf{complete} and \textbf{imputed}. Both gave the same
conclusions, based on our LASSO we dcided to drop 5 variables
(visit\_age, gender, race, weight\_kg, hr)

\pandocbounded{\includegraphics[keepaspectratio]{RegressionFinalProject_files/figure-latex/VariableTransformationsAndBestSubsetPlot-1.pdf}}

Given that our ideal choice of lambda (the one that minimizes cross
validation error) includes all 19 variables selection from the best
subset procedure from previous \textbf{code}. We decided to use
\(5 = -\ln{\lambda}\) as it is at this point where our coefficients
begain to stabilize and our cross validation error does not explode.\\
\textbf{Note:} Lasso results were consistent across complete and imputed
data; plots shown are from complete data.

\pandocbounded{\includegraphics[keepaspectratio]{RegressionFinalProject_files/figure-latex/LassoComplete-1.pdf}}

After preforming LASSO variable selection to reduce multicolinearity,
VIF is mostly under control. The only terms the exhibit a high amount of
multicolinearity are those quadratic terms.

\begin{table}[!h]
\centering
\caption{\label{tab:VIFCheck}VIF Decreasing for both Datasets}
\centering
\begin{tabular}[t]{l|r|r}
\hline
term & GVIF\_complete & GVIF\_imputed\\
\hline
I(insp\_meanatt\textasciicircum{}2) & 867.555 & 1023.801\\
\hline
insp\_meanatt & 838.590 & 992.257\\
\hline
I(exp\_meanatt\textasciicircum{}2) & 515.015 & 884.901\\
\hline
exp\_meanatt & 427.119 & 692.826\\
\hline
\end{tabular}
\end{table}

\subsection{Statsitical Analysis}\label{statsitical-analysis}

\subsubsection{A. Residual Analysis}\label{a.-residual-analysis}

At this point we have a pretty good idea on what variables should be
included in our regression model (including their transformations). Yet,
before we can draw any inferences we need to make sure that our model
meets the 4 major assumptions of Ordinary Least Squares (Zero Mean,
Constant Variance, uncorrelated errors and normality of errors).

\pandocbounded{\includegraphics[keepaspectratio]{RegressionFinalProject_files/figure-latex/olsAssumptions-1.pdf}}
\newpage
\pandocbounded{\includegraphics[keepaspectratio]{RegressionFinalProject_files/figure-latex/olsAssumptions2-1.pdf}}

Although the two plots are constructed from different datasets, they
point to the same conclusions. The assumption of constant variance is
clearly violated. More notably, the normality assumption is strongly
violated. In both tails, the residuals deviate substantially from the
theoretical quantiles, showing pronounced departures from normality at
both the lower and upper extremes. After further investigation we found
the culprit.\\
Both datasets contain imputed values, particularly the \textbf{complete}
dataset. In many instances we imputed rows with multiple missing
entries, sometimes as many as 12. Given the substantial amount of
imputation required, it is likely that some of the imputed values were
inaccurate, and this may have adversely affected the performance of our
linear regression model.\\
Our further analysis analysiszed outliers for both data sets and
determined that in both cases all outliers came from imputed data
(erroneous observations where more pronunced in the \textbf{complete}
dataset). Thus said, we felt that it was appropriate to delete these
erroneously computed observations. More detail on how we determined
``erroneous'' observations in the appendix.\\
After deleting those problematic residuals, we refit our model and plot
our diagnostic plots. As you can observe in both cases our 2 violated
assumptions are almost corrected. Our residuals are now normally
distributed and our variance is now more consistent across all ranges of
fitted values.

\pandocbounded{\includegraphics[keepaspectratio]{RegressionFinalProject_files/figure-latex/ResidualPlotsCombined-1.pdf}}

\subsubsection{B. Model Description, Inference, and
Interpretation}\label{b.-model-description-inference-and-interpretation}

\subsubsection{i. Model Description}\label{i.-model-description}

Our final model was chosen through a multi-step procedure. We first
identified the predictor set with the lowest AIC, then applied LASSO
regularization to remove weak or redundant variables, and finally
excluded a small number of observations with erroneous imputations.\\
Both the complete and imputed datasets produced strong models, with
\(R^2\) values of 0.8600341 and 0.9051658, and adjusted \(R^2\) values
of 0.8596224 and 0.9048315, respectively. At \(\alpha = 0.05\), both
models are statistically significant according to their F-tests,
indicating that the predictors collectively explain substantial variance
in the outcome even though not all individual predictors are
significant.

\begin{table}[!h]
\centering
\caption{\label{tab:CompactModelSummary}Compact Summary of Model Fit Statistics}
\centering
\begin{tabular}[t]{l|r|r|r|r|r|r|r|r|r}
\hline
Model & Residual\_SD & MSE & R2 & Adj\_R2 & F\_stat & df1 & df2 & F\_pvalue & AIC\\
\hline
Complete & 0.140 & 0.020 & 0.860 & 0.860 & 2089.2 & 16 & 5440 & 0 & -5951.8\\
\hline
Imputed & 0.118 & 0.014 & 0.905 & 0.905 & 2707.1 & 16 & 4538 & 0 & -6555.6\\
\hline
\end{tabular}
\end{table}

\subsubsection{ii. Model Discription and
Interpretation}\label{ii.-model-discription-and-interpretation}

Let X1 denote hours of supplemental oxygen used per day; X2 indicate hay
fever; X3 indicate emphysema; X4 indicate COPD; X5 represent average
cigarettes smoked per day; X6 the duration of smoking in years; X7 and
X8 the smoking status indicators; X9 the functional residual capacity;
X10 the percentage of gas trapping; X11 and X12 the inspiratory mean
attenuation and its square; X13 and X14 the expiratory mean attenuation
and its square; X15 the FEV1/FVC ratio; and X16 the FVC value. Let Y
denote the percentage of emphysema. Then the full model is:

\begin{longtable}[]{@{}
  >{\raggedright\arraybackslash}p{(\linewidth - 4\tabcolsep) * \real{0.2895}}
  >{\raggedright\arraybackslash}p{(\linewidth - 4\tabcolsep) * \real{0.4342}}
  >{\raggedright\arraybackslash}p{(\linewidth - 4\tabcolsep) * \real{0.2763}}@{}}
\caption{Predictor Naming Table (3 Predictors per
Column)}\tabularnewline
\toprule\noalign{}
\endfirsthead
\endhead
\bottomrule\noalign{}
\endlastfoot
x1: O2\_hours\_day & x2: hay\_fever & x3: emphysema \\
x4: copd & x5: CigPerDaySmokAvg & x6: Duration\_Smoking \\
x7: smoking\_status & x8: functional\_residual\_capacity & x9:
pct\_gastrapping \\
x10: insp\_meanatt & x11: I(insp\_meanatt\^{}2) & x12: exp\_meanatt \\
x13: I(exp\_meanatt\^{}2) & x14: FEV1\_FVC\_ratio & x15: FVC \\
\end{longtable}

\[ Y = β0 + X1β1 + X2β2 + … + X16β16 + ε \]

Our hypotheses for global inference are the following:\\
\(\text{H0 : All } \beta_i = 0\)\\
\(\text{Ha : At least one } \beta_i \ne 0\)\\
\(\text{ for i = 1, 2, …, 16. }\)

The table summarizes coefficient estimates for 16 variables using both
complete-case (C) and imputed (I) analyses, each paired with 95\%
confidence intervals. Across variables, the imputed estimates generally
track closely with the complete-case results, though some values show
modest shifts in magnitude or interval range. Overall, the comparison
illustrates how imputing missing data can slightly adjust coefficient
estimates while maintaining similar patterns and interpretive
conclusions. We did not control for FWER due to the large number of
predictors included.

\begin{table}[!h]
\centering
\caption{\label{tab:betaTables}Complete-case (C) and imputed (I) coefficient estimates with 95\% confidence intervals.}
\centering
\fontsize{8}{10}\selectfont
\begin{tabular}[t]{rrrrrrr}
\toprule
\$i\$ & \$\textbackslash{}beta\_i\textasciicircum{}\{(C)\}\$ & \$L\_\{95\}\textasciicircum{}\{(C)\}\$ & \$U\_\{95\}\textasciicircum{}\{(C)\}\$ & \$\textbackslash{}beta\_i\textasciicircum{}\{(I)\}\$ & \$L\_\{95\}\textasciicircum{}\{(I)\}\$ & \$U\_\{95\}\textasciicircum{}\{(I)\}\$\\
\midrule
1 & 0.002 & 0.001 & 0.003 & 0.002 & 0.001 & 0.003\\
2 & -0.006 & -0.014 & 0.002 & -0.008 & -0.016 & -0.001\\
3 & 0.019 & 0.007 & 0.032 & 0.013 & 0.001 & 0.024\\
4 & 0.015 & 0.003 & 0.027 & 0.014 & 0.003 & 0.025\\
5 & 0.001 & 0.001 & 0.002 & 0.001 & 0.001 & 0.001\\
\addlinespace
6 & 0.001 & 0.000 & 0.001 & 0.000 & 0.000 & 0.001\\
7 & 0.019 & 0.010 & 0.028 & -0.002 & -0.011 & 0.006\\
8 & -0.028 & -0.065 & 0.010 & -0.040 & -0.073 & -0.007\\
9 & -0.032 & -0.039 & -0.024 & -0.011 & -0.019 & -0.003\\
10 & 0.022 & 0.021 & 0.023 & 0.031 & 0.030 & 0.031\\
\addlinespace
11 & 0.037 & 0.034 & 0.040 & 0.044 & 0.041 & 0.048\\
12 & 0.000 & 0.000 & 0.000 & 0.000 & 0.000 & 0.000\\
13 & -0.017 & -0.019 & -0.016 & -0.029 & -0.031 & -0.028\\
14 & 0.000 & 0.000 & 0.000 & 0.000 & 0.000 & 0.000\\
15 & -0.311 & -0.361 & -0.262 & -0.293 & -0.340 & -0.246\\
\addlinespace
16 & 0.015 & 0.009 & 0.021 & -0.004 & -0.010 & 0.002\\
\bottomrule
\end{tabular}
\end{table}

\subsubsection{Conclusion}\label{conclusion}

In summary, our analysis identifies a clear set of health related
variables that play the strongest role in predicting the percentage of
emphysema present in a patient's lungs. Measures tied directly to
respiratory function, such as lung capacity metrics, gas trapping
indicators, and attenuation based imaging variables, emerged as the most
influential predictors. This aligns with expectations, as these metrics
reflect the structural and functional deterioration of lung tissue.
Prior diagnoses of COPD and emphysema further strengthened the
predictive power of the model, reinforcing the progressive nature of
respiratory illness and its measurable impact on lung health.

Smoking behavior, including long term exposure and daily quantity, also
contributed meaningfully to the prediction of emphysema percentage.
While smoking is widely recognized as a primary cause of emphysema, our
regression results confirm that smoking related variables are
consistently associated with higher emphysema severity. In contrast,
several general health indicators such as BMI, heart rate, and blood
pressure were not as impactful as originally hypothesized, suggesting
that emphysema severity is driven more by respiratory specific factors
than by broader measures of overall health.

As with any statistical study, limitations must be acknowledged. The
dataset reflects health measurements at a single point in time, even
though emphysema develops progressively. Like mentioned in the
introduction linear regression also assumes additive relationships
between predictors, which may oversimplify the complex interactions
underlying lung deterioration. Additionally, the need to impute missing
values in both datasets reduced data quality and led to the deletion of
several observations, ultimately making the model less powerful.
Nevertheless, linear regression remains valuable in this context because
it offers clear interpretability, making it well suited for identifying
which variables contribute most strongly to emphysema percentage.

While this study confirms linear regression's utility as a transparent
and methodologically sound first step in emphysema research, it also
highlights opportunities for methodological advancement. Future
investigations should incorporate longitudinal data to model temporal
dynamics and employ nonlinear approaches (e.g., generalized additive
models, machine learning techniques) to capture the complex interactions
inherent in respiratory pathophysiology. These methodological expansions
would complement---not invalidate---the foundational relationships
identified through linear regression, creating a more comprehensive
understanding of emphysema progression.

\newpage

\subsection{Refrences}\label{refrences}

\begin{enumerate}
\def\labelenumi{\arabic{enumi}.}
\item
  Mayo Clinic Staff. Emphysema: Symptoms and Causes. Mayo Clinic.
  \url{https://www.mayoclinic.org/diseases-conditions/emphysema/symptoms-causes/syc-20355555}.
\item
  Cleveland Clinic Staff. Emphysema: Causes, Symptoms, and Treatment.
  Cleveland Clinic.
  \url{https://my.clevelandclinic.org/health/diseases/9370-emphysema}.
\item
  Hasenstab, K. Lecture Slides on Linear Models, Regression Diagnostics,
  and Variable Selection. Department of Statistics, {[}San Diego State
  University{]}, 2025.
\end{enumerate}

\newpage

\subsection{Appendix}\label{appendix}

We will start our appendix by describing how our data was preprocessed.

\begin{Shaded}
\begin{Highlighting}[]
\CommentTok{\# Load Data {-}{-}{-}{-}{-}{-}{-}{-}{-}{-}{-}{-}{-}{-}{-}{-}{-}{-}{-}{-}{-}{-}{-}{-}{-}{-}{-}{-}{-}{-}{-}{-}{-}{-}{-}{-}{-}{-}{-}{-}{-}{-}{-}{-}{-}{-}{-}{-}{-}{-}{-}{-}{-}{-}{-}{-}{-}{-}{-}{-}{-}{-}{-}}

\CommentTok{\# Read the COPD dataset from a CSV file into a data frame}
\NormalTok{df\_copd }\OtherTok{=} \FunctionTok{read\_csv}\NormalTok{( }\StringTok{"/cloud/project/copd\_data.csv"}\NormalTok{ )}


\CommentTok{\# Replace {-}1\textquotesingle{}s with NA\textquotesingle{}s {-}{-}{-}{-}{-}{-}{-}{-}{-}{-}{-}{-}{-}{-}{-}{-}{-}{-}{-}{-}{-}{-}{-}{-}{-}{-}{-}{-}{-}{-}{-}{-}{-}{-}{-}{-}{-}{-}{-}{-}{-}{-}{-}{-}{-}{-}{-}{-}{-}{-}}

\CommentTok{\# For various continuous / numeric variables, recode negative sentinel values}
\CommentTok{\# (e.g., {-}1) to NA to represent missingness.}
\NormalTok{df\_copd }\OtherTok{=}\NormalTok{ df\_copd }\SpecialCharTok{|\textgreater{}}
  \FunctionTok{mutate}\NormalTok{( }
    \CommentTok{\# Systolic blood pressure: set negative values to NA}
    \AttributeTok{sysBP =} \FunctionTok{ifelse}\NormalTok{( sysBP }\SpecialCharTok{\textless{}} \DecValTok{0}\NormalTok{, }\ConstantTok{NA}\NormalTok{, sysBP ),}
    \CommentTok{\# Diastolic blood pressure: set negative values to NA}
    \AttributeTok{diasBP =} \FunctionTok{ifelse}\NormalTok{( diasBP }\SpecialCharTok{\textless{}} \DecValTok{0}\NormalTok{, }\ConstantTok{NA}\NormalTok{, diasBP ), }
    \CommentTok{\# Heart rate: set negative values to NA}
    \AttributeTok{hr =} \FunctionTok{ifelse}\NormalTok{( hr }\SpecialCharTok{\textless{}} \DecValTok{0}\NormalTok{, }\ConstantTok{NA}\NormalTok{, hr ), }
    \CommentTok{\# Recode hay\_fever numeric codes into categorical labels}
    \AttributeTok{hay\_fever =} \FunctionTok{case\_when}\NormalTok{(}
\NormalTok{      hay\_fever }\SpecialCharTok{==} \DecValTok{0} \SpecialCharTok{\textasciitilde{}} \StringTok{"No"}\NormalTok{,}
\NormalTok{      hay\_fever }\SpecialCharTok{==} \DecValTok{1} \SpecialCharTok{\textasciitilde{}} \StringTok{"Yes"}\NormalTok{,}
\NormalTok{      hay\_fever }\SpecialCharTok{==} \DecValTok{3} \SpecialCharTok{\textasciitilde{}} \StringTok{"unknown"}
\NormalTok{    ),}
    \CommentTok{\# Smoking{-}related variables: convert {-}1 sentinel to NA}
    \AttributeTok{SmokStartAge =} \FunctionTok{ifelse}\NormalTok{( SmokStartAge }\SpecialCharTok{==} \SpecialCharTok{{-}}\DecValTok{1}\NormalTok{ , }\ConstantTok{NA}\NormalTok{, SmokStartAge),}
    \AttributeTok{CigPerDaySmokAvg =} \FunctionTok{ifelse}\NormalTok{( CigPerDaySmokAvg }\SpecialCharTok{==} \SpecialCharTok{{-}}\DecValTok{1}\NormalTok{, }\ConstantTok{NA}\NormalTok{, CigPerDaySmokAvg),}
    \AttributeTok{Duration\_Smoking =} \FunctionTok{ifelse}\NormalTok{( Duration\_Smoking }\SpecialCharTok{==} \SpecialCharTok{{-}}\DecValTok{1}\NormalTok{, }\ConstantTok{NA}\NormalTok{, Duration\_Smoking),}
    \CommentTok{\# Lung function / imaging measures: convert {-}1 sentinel to NA}
    \AttributeTok{total\_lung\_capacity =} \FunctionTok{ifelse}\NormalTok{( total\_lung\_capacity }\SpecialCharTok{==} \SpecialCharTok{{-}}\DecValTok{1}\NormalTok{ , }\ConstantTok{NA}\NormalTok{, total\_lung\_capacity),}
    \AttributeTok{pct\_emphysema =} \FunctionTok{ifelse}\NormalTok{( pct\_emphysema }\SpecialCharTok{==} \SpecialCharTok{{-}}\DecValTok{1}\NormalTok{ , }\ConstantTok{NA}\NormalTok{, pct\_emphysema),}
    \AttributeTok{functional\_residual\_capacity =} \FunctionTok{ifelse}\NormalTok{( functional\_residual\_capacity }\SpecialCharTok{==} \SpecialCharTok{{-}}\DecValTok{1}\NormalTok{ , }\ConstantTok{NA}\NormalTok{, functional\_residual\_capacity),}
    \AttributeTok{pct\_gastrapping =} \FunctionTok{ifelse}\NormalTok{( pct\_gastrapping }\SpecialCharTok{==} \SpecialCharTok{{-}}\DecValTok{1}\NormalTok{ , }\ConstantTok{NA}\NormalTok{, pct\_gastrapping),}
    \AttributeTok{insp\_meanatt =} \FunctionTok{ifelse}\NormalTok{( insp\_meanatt }\SpecialCharTok{==} \SpecialCharTok{{-}}\DecValTok{1}\NormalTok{, }\ConstantTok{NA}\NormalTok{, insp\_meanatt),}
    \AttributeTok{exp\_meanatt =} \FunctionTok{ifelse}\NormalTok{( exp\_meanatt }\SpecialCharTok{==} \SpecialCharTok{{-}}\DecValTok{1}\NormalTok{, }\ConstantTok{NA}\NormalTok{, exp\_meanatt),}
    \AttributeTok{FEV1\_FVC\_ratio =} \FunctionTok{ifelse}\NormalTok{( FEV1\_FVC\_ratio }\SpecialCharTok{==} \SpecialCharTok{{-}}\DecValTok{1}\NormalTok{, }\ConstantTok{NA}\NormalTok{, FEV1\_FVC\_ratio),}
    \AttributeTok{FEV1 =} \FunctionTok{ifelse}\NormalTok{( FEV1 }\SpecialCharTok{==} \SpecialCharTok{{-}}\DecValTok{1}\NormalTok{, }\ConstantTok{NA}\NormalTok{, FEV1),}
    \AttributeTok{FVC =} \FunctionTok{ifelse}\NormalTok{( FVC }\SpecialCharTok{==} \SpecialCharTok{{-}}\DecValTok{1}\NormalTok{, }\ConstantTok{NA}\NormalTok{, FVC)}
\NormalTok{  )}


\CommentTok{\# Convert categorical "unknown" to NA\textquotesingle{}s {-}{-}{-}{-}{-}{-}{-}{-}{-}{-}{-}{-}{-}{-}{-}{-}{-}{-}{-}{-}{-}{-}{-}{-}{-}{-}{-}{-}{-}{-}{-}{-}{-}{-}{-}}

\CommentTok{\# For categorical variables, convert the "unknown" (and "missing" for sleep\_apnea)}
\CommentTok{\# categories into NA to treat them as missing.}
\NormalTok{df\_copd }\OtherTok{=}\NormalTok{ df\_copd }\SpecialCharTok{|\textgreater{}} \FunctionTok{mutate}\NormalTok{(}
  \AttributeTok{asthma =} \FunctionTok{ifelse}\NormalTok{( asthma }\SpecialCharTok{==} \StringTok{"unknown"}\NormalTok{, }\ConstantTok{NA}\NormalTok{, asthma),}
  \AttributeTok{hay\_fever =} \FunctionTok{ifelse}\NormalTok{( hay\_fever }\SpecialCharTok{==} \StringTok{"unknown"}\NormalTok{, }\ConstantTok{NA}\NormalTok{, hay\_fever),}
  \AttributeTok{bronchitis\_attack =} \FunctionTok{ifelse}\NormalTok{( bronchitis\_attack }\SpecialCharTok{==} \StringTok{"unknown"}\NormalTok{, }\ConstantTok{NA}\NormalTok{, bronchitis\_attack),}
  \AttributeTok{pneumonia =} \FunctionTok{ifelse}\NormalTok{( pneumonia }\SpecialCharTok{==} \StringTok{"unknown"}\NormalTok{, }\ConstantTok{NA}\NormalTok{, pneumonia),}
  \AttributeTok{chronic\_bronchitis =} \FunctionTok{ifelse}\NormalTok{( chronic\_bronchitis }\SpecialCharTok{==} \StringTok{"unknown"}\NormalTok{, }\ConstantTok{NA}\NormalTok{, chronic\_bronchitis),}
  \AttributeTok{emphysema =} \FunctionTok{ifelse}\NormalTok{( emphysema }\SpecialCharTok{==} \StringTok{"unknown"}\NormalTok{, }\ConstantTok{NA}\NormalTok{, emphysema),}
  \AttributeTok{copd =} \FunctionTok{ifelse}\NormalTok{( copd }\SpecialCharTok{==} \StringTok{"unknown"}\NormalTok{, }\ConstantTok{NA}\NormalTok{, copd),}
  \CommentTok{\# Sleep apnea: treat both "unknown" and "missing" as NA}
  \AttributeTok{sleep\_apnea =} \FunctionTok{ifelse}\NormalTok{( (sleep\_apnea }\SpecialCharTok{==} \StringTok{"unknown"}\NormalTok{) }\SpecialCharTok{|}\NormalTok{ (sleep\_apnea }\SpecialCharTok{==} \StringTok{"missing"}\NormalTok{) , }\ConstantTok{NA}\NormalTok{, sleep\_apnea)}
\NormalTok{)}


\CommentTok{\# Convert categorical to binary {-}{-}{-}{-}{-}{-}{-}{-}{-}{-}{-}{-}{-}{-}{-}{-}{-}{-}{-}{-}{-}{-}{-}{-}{-}{-}{-}{-}{-}{-}{-}{-}{-}{-}{-}{-}{-}{-}{-}{-}{-}{-}{-}}

\CommentTok{\# Convert Yes/No categorical variables into numeric 0/1 indicators,}
\CommentTok{\# preserving NA where values are missing.}
\NormalTok{df\_copd }\OtherTok{\textless{}{-}}\NormalTok{ df\_copd }\SpecialCharTok{|\textgreater{}} 
  \FunctionTok{mutate}\NormalTok{(}
    \AttributeTok{asthma =} \FunctionTok{case\_when}\NormalTok{(}
      \FunctionTok{is.na}\NormalTok{(asthma) }\SpecialCharTok{\textasciitilde{}} \ConstantTok{NA\_real\_}\NormalTok{,  }\CommentTok{\# keep missing as NA}
\NormalTok{      asthma }\SpecialCharTok{==} \StringTok{"No"} \SpecialCharTok{\textasciitilde{}} \DecValTok{0}\NormalTok{,}
\NormalTok{      asthma }\SpecialCharTok{==} \StringTok{"Yes"} \SpecialCharTok{\textasciitilde{}} \DecValTok{1}
\NormalTok{    ),}
    \AttributeTok{hay\_fever =} \FunctionTok{case\_when}\NormalTok{(}
      \FunctionTok{is.na}\NormalTok{(hay\_fever) }\SpecialCharTok{\textasciitilde{}} \ConstantTok{NA\_real\_}\NormalTok{,}
\NormalTok{      hay\_fever }\SpecialCharTok{==} \StringTok{"No"} \SpecialCharTok{\textasciitilde{}} \DecValTok{0}\NormalTok{,}
\NormalTok{      hay\_fever }\SpecialCharTok{==} \StringTok{"Yes"} \SpecialCharTok{\textasciitilde{}} \DecValTok{1}
\NormalTok{    ),}
    \AttributeTok{bronchitis\_attack =} \FunctionTok{case\_when}\NormalTok{(}
      \FunctionTok{is.na}\NormalTok{(bronchitis\_attack) }\SpecialCharTok{\textasciitilde{}} \ConstantTok{NA\_real\_}\NormalTok{,}
\NormalTok{      bronchitis\_attack }\SpecialCharTok{==} \StringTok{"No"} \SpecialCharTok{\textasciitilde{}} \DecValTok{0}\NormalTok{,}
\NormalTok{      bronchitis\_attack }\SpecialCharTok{==} \StringTok{"Yes"} \SpecialCharTok{\textasciitilde{}} \DecValTok{1}
\NormalTok{    ),}
    \AttributeTok{pneumonia =} \FunctionTok{case\_when}\NormalTok{(}
      \FunctionTok{is.na}\NormalTok{(pneumonia) }\SpecialCharTok{\textasciitilde{}} \ConstantTok{NA\_real\_}\NormalTok{,}
\NormalTok{      pneumonia }\SpecialCharTok{==} \StringTok{"No"} \SpecialCharTok{\textasciitilde{}} \DecValTok{0}\NormalTok{,}
\NormalTok{      pneumonia }\SpecialCharTok{==} \StringTok{"Yes"} \SpecialCharTok{\textasciitilde{}} \DecValTok{1}
\NormalTok{    ),}
    \AttributeTok{chronic\_bronchitis =} \FunctionTok{case\_when}\NormalTok{(}
      \FunctionTok{is.na}\NormalTok{(chronic\_bronchitis) }\SpecialCharTok{\textasciitilde{}} \ConstantTok{NA\_real\_}\NormalTok{,}
\NormalTok{      chronic\_bronchitis }\SpecialCharTok{==} \StringTok{"No"} \SpecialCharTok{\textasciitilde{}} \DecValTok{0}\NormalTok{,}
\NormalTok{      chronic\_bronchitis }\SpecialCharTok{==} \StringTok{"Yes"} \SpecialCharTok{\textasciitilde{}} \DecValTok{1}
\NormalTok{    ),}
    \AttributeTok{emphysema =} \FunctionTok{case\_when}\NormalTok{(}
      \FunctionTok{is.na}\NormalTok{(emphysema) }\SpecialCharTok{\textasciitilde{}} \ConstantTok{NA\_real\_}\NormalTok{,}
\NormalTok{      emphysema }\SpecialCharTok{==} \StringTok{"No"} \SpecialCharTok{\textasciitilde{}} \DecValTok{0}\NormalTok{,}
\NormalTok{      emphysema }\SpecialCharTok{==} \StringTok{"Yes"} \SpecialCharTok{\textasciitilde{}} \DecValTok{1}
\NormalTok{    ),}
    \AttributeTok{copd =} \FunctionTok{case\_when}\NormalTok{(}
      \FunctionTok{is.na}\NormalTok{(copd) }\SpecialCharTok{\textasciitilde{}} \ConstantTok{NA\_real\_}\NormalTok{,}
\NormalTok{      copd }\SpecialCharTok{==} \StringTok{"No"} \SpecialCharTok{\textasciitilde{}} \DecValTok{0}\NormalTok{,}
\NormalTok{      copd }\SpecialCharTok{==} \StringTok{"Yes"} \SpecialCharTok{\textasciitilde{}} \DecValTok{1}
\NormalTok{    ),}
    \AttributeTok{sleep\_apnea =} \FunctionTok{case\_when}\NormalTok{(}
      \FunctionTok{is.na}\NormalTok{(sleep\_apnea) }\SpecialCharTok{\textasciitilde{}} \ConstantTok{NA\_real\_}\NormalTok{,}
\NormalTok{      sleep\_apnea }\SpecialCharTok{==} \StringTok{"No"} \SpecialCharTok{\textasciitilde{}} \DecValTok{0}\NormalTok{,}
\NormalTok{      sleep\_apnea }\SpecialCharTok{==} \StringTok{"Yes"} \SpecialCharTok{\textasciitilde{}} \DecValTok{1}
\NormalTok{    )}
\NormalTok{  )}


\CommentTok{\# Convert into factors {-}{-}{-}{-}{-}{-}{-}{-}{-}{-}{-}{-}{-}{-}{-}{-}{-}{-}{-}{-}{-}{-}{-}{-}{-}{-}{-}{-}{-}{-}{-}{-}{-}{-}{-}{-}{-}{-}{-}{-}{-}{-}{-}{-}{-}{-}{-}{-}{-}{-}{-}{-}}

\CommentTok{\# Coerce selected variables to factors for categorical analysis/models.}
\NormalTok{df\_copd }\OtherTok{=}\NormalTok{ df\_copd }\SpecialCharTok{|\textgreater{}}
  \FunctionTok{mutate}\NormalTok{( }
    \AttributeTok{gender =} \FunctionTok{as.factor}\NormalTok{(gender),}
    \AttributeTok{race =} \FunctionTok{as.factor}\NormalTok{(race),}
    \AttributeTok{asthma =} \FunctionTok{as.factor}\NormalTok{(asthma),}
    \AttributeTok{hay\_fever =} \FunctionTok{as.factor}\NormalTok{(hay\_fever),}
    \AttributeTok{bronchitis\_attack =} \FunctionTok{as.factor}\NormalTok{(bronchitis\_attack),}
    \AttributeTok{pneumonia =} \FunctionTok{as.factor}\NormalTok{(pneumonia),}
    \AttributeTok{chronic\_bronchitis =} \FunctionTok{as.factor}\NormalTok{(chronic\_bronchitis),}
    \AttributeTok{emphysema =} \FunctionTok{as.factor}\NormalTok{(emphysema),}
    \AttributeTok{copd =} \FunctionTok{as.factor}\NormalTok{(copd),}
    \AttributeTok{sleep\_apnea =} \FunctionTok{as.factor}\NormalTok{(sleep\_apnea),}
    \AttributeTok{smoking\_status =} \FunctionTok{as.factor}\NormalTok{(smoking\_status),}
    \AttributeTok{gender =} \FunctionTok{as.factor}\NormalTok{(gender)  }\CommentTok{\# redundant re{-}coercion of gender to factor}
\NormalTok{  )}

\CommentTok{\# Relocate percent of emphysema {-}{-}{-}{-}{-}{-}{-}{-}{-}{-}{-}{-}{-}{-}{-}{-}{-}{-}{-}{-}{-}{-}{-}{-}{-}{-}{-}{-}{-}{-}{-}{-}{-}{-}{-}{-}{-}{-}{-}{-}{-}{-}}

\CommentTok{\# Move pct\_emphysema column to immediately follow sid in the data frame.}
\NormalTok{df\_copd }\OtherTok{=}\NormalTok{ df\_copd }\SpecialCharTok{|\textgreater{}}
  \FunctionTok{relocate}\NormalTok{( pct\_emphysema, }\AttributeTok{.after =}\NormalTok{ sid )}

\CommentTok{\# Zeros for smoke age and CigPerDaySmokAvg {-}{-}{-}{-}{-}{-}{-}{-}{-}{-}{-}{-}{-}{-}{-}{-}{-}{-}{-}{-}{-}{-}{-}{-}{-}{-}{-}{-}{-}{-}{-}}

\CommentTok{\# For non{-}smokers ("Never smoked"), set smoking{-}related measures to 0}
\CommentTok{\# instead of leaving them as NA or original values.}
\NormalTok{df\_copd }\OtherTok{\textless{}{-}}\NormalTok{ df\_copd }\SpecialCharTok{|\textgreater{}}
  \FunctionTok{mutate}\NormalTok{(}
    \AttributeTok{SmokStartAge =} \FunctionTok{if\_else}\NormalTok{(smoking\_status }\SpecialCharTok{==} \StringTok{"Never smoked"}\NormalTok{, }\DecValTok{0}\NormalTok{, SmokStartAge),}
    \AttributeTok{CigPerDaySmokAvg =} \FunctionTok{if\_else}\NormalTok{(smoking\_status }\SpecialCharTok{==} \StringTok{"Never smoked"}\NormalTok{, }\DecValTok{0}\NormalTok{, CigPerDaySmokAvg),}
    \AttributeTok{Duration\_Smoking =} \FunctionTok{if\_else}\NormalTok{(smoking\_status }\SpecialCharTok{==} \StringTok{"Never smoked"}\NormalTok{, }\DecValTok{0}\NormalTok{, Duration\_Smoking)}
\NormalTok{  )}
\end{Highlighting}
\end{Shaded}

The subsequent sections of the appendix provide supplementary material,
organized to mirror the structure of the main analysis.

\newpage

\subsection{Explination of variables
used}\label{explination-of-variables-used}

Below are the original 35 variables as well as their description (not
all where used or relevant in the analysis):

• sid: The anonymized patient identification number.\\
• visit\_year: The calendar year in which the patient visit occurred.\\
• visit\_date: The specific date on which the patient visit occurred.\\
• visit\_age: The patient's age at the time of the visit.\\
• gender: The patient's reported gender (Male or Female).\\
• race: The patient's race category (White, Black or African
American).\\
• height\_cm: The patient's height measured in centimeters.\\
• weight\_kg: The patient's weight measured in kilograms.\\
• sysBP, diasBP: Systolic and diastolic blood pressure, respectively.\\
• hr: The patient's heart rate.\\
• O2\_hours\_day: For a typical 24-hour day, the number of hours of
supplemental oxygen used.\\
• bmi: The patient's body mass index.\\
• asthma: Whether the patient has ever been diagnosed with asthma (Yes,
No).\\
• hay\_fever: Whether the patient has ever had hay fever (Yes, No).\\
• bronchitis\_attack: Whether the patient has ever had a bronchitis
attack (Yes, No).\\
• pneumonia: Whether the patient has ever had pneumonia (Yes, No).\\
• chronic\_bronchitis: Whether the patient has ever been diagnosed with
chronic bronchitis (Yes, No).\\
• emphysema: Whether the patient has ever been diagnosed with emphysema
(Yes, No).\\
• copd: Whether the patient has been diagnosed with chronic obstructive
pulmonary disease (Yes, No).\\
• sleep\_apnea: Whether the patient has ever had sleep apnea (Yes,
No).\\
• SmokStartAge: The age at which the patient began cigarette smoking.\\
• CigPerDaySmokAvg: The average number of cigarettes smoked per day
across smoking history.\\
• Duration\_Smoking: The number of years the patient has smoked.\\
• smoking\_status: Categorical indicator of smoking behavior (Never
smoked, Former smoker, Current smoker).\\
• total\_lung\_capacity: The lung volume at full inspiration, measured
in liters.\\
• pct\_emphysema: The percentage of emphysematous (damaged) lung
tissue.\\
• functional\_residual\_capacity: The volume of air remaining in the
lungs at the end of expiration, in liters.\\
• pct\_gastrapping: The percentage of air trapping present after
exhalation.\\
• insp\_meanatt: The average lung density at full inspiration, measured
in Hounsfield units.\\
• exp\_meanatt: The average lung density at expiration, measured in
Hounsfield units.\\
• FEV1\_FVC\_ratio: The ratio between forced expiratory volume in 1
second (FEV1) and forced vital capacity (FVC).\\
• FEV1: The volume of air forcefully exhaled in 1 second.\\
• FVC: The total exhaled air volume after a full inhalation.\\
• FEV1\_phase2: The FEV1 value measured five years later during a
follow-up assessment.

\newpage

\subsection{Full table summaries and their
distributions}\label{full-table-summaries-and-their-distributions}

Originally, due to the lack of space in the study and lack of relevance
of some variables most variables where excluded from our table summaries
and their distributions where not displayed.

\begin{verbatim}
## # A tibble: 35 x 8
##    variable      unique missing    mean     sd         min  median    max
##    <chr>          <dbl>   <dbl>   <dbl>  <dbl>       <dbl>   <dbl>  <dbl>
##  1 sid             5747       0   NA    NA       NA          NA      NA  
##  2 pct_emphysema   4698    1045    5.58  8.43     0.000196    2.11   60.6
##  3 visit_year         4       0 2009.    0.808 2008        2009    2011  
##  4 visit_date        40       0   NA    NA       NA          NA      NA  
##  5 visit_age        392       0   59.7   8.69    39          59.5    85  
##  6 gender             2       0   NA    NA       NA          NA      NA  
##  7 race               2       0   NA    NA       NA          NA      NA  
##  8 height_cm        437       0  170.    9.53   134.        170     208. 
##  9 weight_kg        833       0   84.1  19.6     34.9        82     176. 
## 10 sysBP            111       2  129.   16.6     80         128     211  
## # i 25 more rows
\end{verbatim}

\subsection{Histograms and Boxplots}\label{histograms-and-boxplots}

\begin{Shaded}
\begin{Highlighting}[]
\CommentTok{\# Use df\_copd as the dataset}
\NormalTok{df\_single }\OtherTok{\textless{}{-}}\NormalTok{ df\_copd }\SpecialCharTok{|\textgreater{}}
\NormalTok{  dplyr}\SpecialCharTok{::}\FunctionTok{select}\NormalTok{( }\SpecialCharTok{{-}}\NormalTok{sid, }\SpecialCharTok{{-}}\NormalTok{visit\_year, }\SpecialCharTok{{-}}\NormalTok{visit\_date )}
\NormalTok{vars }\OtherTok{\textless{}{-}} \FunctionTok{names}\NormalTok{(df\_single)}

\CommentTok{\# Ensure pct\_emphysema exists}
\ControlFlowTok{if}\NormalTok{ (}\SpecialCharTok{!}\StringTok{"pct\_emphysema"} \SpecialCharTok{\%in\%}\NormalTok{ vars) \{}
  \FunctionTok{stop}\NormalTok{(}\StringTok{"pct\_emphysema not found in df\_copd."}\NormalTok{)}
\NormalTok{\}}

\ControlFlowTok{for}\NormalTok{ (v }\ControlFlowTok{in}\NormalTok{ vars) \{}

  \FunctionTok{cat}\NormalTok{(}\StringTok{"}\SpecialCharTok{\textbackslash{}n\textbackslash{}n}\StringTok{\#\#\# Plotting:"}\NormalTok{, v, }\StringTok{"\#\#\#}\SpecialCharTok{\textbackslash{}n}\StringTok{"}\NormalTok{)}

  \CommentTok{\# {-}{-}{-}{-}{-}{-}{-}{-}{-}{-}{-}{-}{-}{-}{-}{-}{-}{-}{-}{-}{-}{-}{-}{-}{-}{-}{-}{-}{-}{-}{-}{-}{-}{-}{-}{-}{-}{-}{-}{-}{-}{-}{-}{-}{-}{-}{-}{-}{-}{-}{-}{-}{-}{-}{-}{-}{-} \#}
  \CommentTok{\#  NUMERIC VARIABLES → HISTOGRAM OF THE VARIABLE ITSELF     \#}
  \CommentTok{\# {-}{-}{-}{-}{-}{-}{-}{-}{-}{-}{-}{-}{-}{-}{-}{-}{-}{-}{-}{-}{-}{-}{-}{-}{-}{-}{-}{-}{-}{-}{-}{-}{-}{-}{-}{-}{-}{-}{-}{-}{-}{-}{-}{-}{-}{-}{-}{-}{-}{-}{-}{-}{-}{-}{-}{-}{-} \#}
  \ControlFlowTok{if}\NormalTok{ (}\FunctionTok{is.numeric}\NormalTok{(df\_single[[v]]) }\SpecialCharTok{\&}\NormalTok{ v }\SpecialCharTok{!=} \StringTok{"pct\_emphysema"}\NormalTok{) \{}

\NormalTok{    p\_hist }\OtherTok{\textless{}{-}} \FunctionTok{ggplot}\NormalTok{(df\_single, }\FunctionTok{aes}\NormalTok{(}\AttributeTok{x =}\NormalTok{ .data[[v]])) }\SpecialCharTok{+}
      \FunctionTok{geom\_histogram}\NormalTok{(}\AttributeTok{bins =} \DecValTok{30}\NormalTok{, }\AttributeTok{fill =} \StringTok{"skyblue"}\NormalTok{, }\AttributeTok{color =} \StringTok{"black"}\NormalTok{) }\SpecialCharTok{+}
      \FunctionTok{labs}\NormalTok{(}
        \AttributeTok{title =} \FunctionTok{paste}\NormalTok{(}\StringTok{"Histogram for"}\NormalTok{, v),}
        \AttributeTok{x =}\NormalTok{ v, }\AttributeTok{y =} \StringTok{"Count"}
\NormalTok{      ) }\SpecialCharTok{+}
      \FunctionTok{theme\_bw}\NormalTok{()}

    \FunctionTok{print}\NormalTok{(p\_hist)}

\NormalTok{  \} }\ControlFlowTok{else} \ControlFlowTok{if}\NormalTok{ (}\SpecialCharTok{!}\FunctionTok{is.numeric}\NormalTok{(df\_single[[v]])) \{}

    \CommentTok{\# {-}{-}{-}{-}{-}{-}{-}{-}{-}{-}{-}{-}{-}{-}{-}{-}{-}{-}{-}{-}{-}{-}{-}{-}{-}{-}{-}{-}{-}{-}{-}{-}{-}{-}{-}{-}{-}{-}{-}{-}{-}{-}{-}{-}{-}{-}{-}{-}{-}{-}{-}{-}{-}{-}{-}{-}{-} \#}
    \CommentTok{\#  CATEGORICAL VARIABLES → BOXPLOT OF pct\_emphysema BY v    \#}
    \CommentTok{\# {-}{-}{-}{-}{-}{-}{-}{-}{-}{-}{-}{-}{-}{-}{-}{-}{-}{-}{-}{-}{-}{-}{-}{-}{-}{-}{-}{-}{-}{-}{-}{-}{-}{-}{-}{-}{-}{-}{-}{-}{-}{-}{-}{-}{-}{-}{-}{-}{-}{-}{-}{-}{-}{-}{-}{-}{-} \#}
\NormalTok{    p\_box }\OtherTok{\textless{}{-}}\NormalTok{ df\_single }\SpecialCharTok{\%\textgreater{}\%}
      \FunctionTok{drop\_na}\NormalTok{(.data[[v]], pct\_emphysema) }\SpecialCharTok{\%\textgreater{}\%}
      \FunctionTok{ggplot}\NormalTok{(}\FunctionTok{aes}\NormalTok{(}\AttributeTok{x =}\NormalTok{ .data[[v]], }\AttributeTok{y =}\NormalTok{ pct\_emphysema)) }\SpecialCharTok{+}
      \FunctionTok{geom\_boxplot}\NormalTok{(}\AttributeTok{fill =} \StringTok{"skyblue"}\NormalTok{, }\AttributeTok{color =} \StringTok{"black"}\NormalTok{, }\AttributeTok{outlier.color =} \StringTok{"black"}\NormalTok{) }\SpecialCharTok{+}
      \FunctionTok{labs}\NormalTok{(}
        \AttributeTok{title =} \FunctionTok{paste}\NormalTok{(}\StringTok{"pct\_emphysema by"}\NormalTok{, v),}
        \AttributeTok{x =}\NormalTok{ v, }\AttributeTok{y =} \StringTok{"pct\_emphysema"}
\NormalTok{      ) }\SpecialCharTok{+}
      \FunctionTok{theme\_bw}\NormalTok{()}

    \FunctionTok{print}\NormalTok{(p\_box)}
\NormalTok{  \}}
\NormalTok{\}}
\end{Highlighting}
\end{Shaded}

\begin{verbatim}
## 
## 
## ### Plotting: pct_emphysema ###
## 
## 
## ### Plotting: visit_age ###
\end{verbatim}

\pandocbounded{\includegraphics[keepaspectratio]{RegressionFinalProject_files/figure-latex/appendixHistograms-1.pdf}}

\begin{verbatim}
## 
## 
## ### Plotting: gender ###
\end{verbatim}

\begin{verbatim}
## Warning: Use of .data in tidyselect expressions was deprecated in tidyselect 1.2.0.
## i Please use `all_of(var)` (or `any_of(var)`) instead of `.data[[var]]`
## This warning is displayed once every 8 hours.
## Call `lifecycle::last_lifecycle_warnings()` to see where this warning was
## generated.
\end{verbatim}

\pandocbounded{\includegraphics[keepaspectratio]{RegressionFinalProject_files/figure-latex/appendixHistograms-2.pdf}}

\begin{verbatim}
## 
## 
## ### Plotting: race ###
\end{verbatim}

\pandocbounded{\includegraphics[keepaspectratio]{RegressionFinalProject_files/figure-latex/appendixHistograms-3.pdf}}

\begin{verbatim}
## 
## 
## ### Plotting: height_cm ###
\end{verbatim}

\pandocbounded{\includegraphics[keepaspectratio]{RegressionFinalProject_files/figure-latex/appendixHistograms-4.pdf}}

\begin{verbatim}
## 
## 
## ### Plotting: weight_kg ###
\end{verbatim}

\pandocbounded{\includegraphics[keepaspectratio]{RegressionFinalProject_files/figure-latex/appendixHistograms-5.pdf}}

\begin{verbatim}
## 
## 
## ### Plotting: sysBP ###
\end{verbatim}

\begin{verbatim}
## Warning: Removed 2 rows containing non-finite outside the scale range
## (`stat_bin()`).
\end{verbatim}

\pandocbounded{\includegraphics[keepaspectratio]{RegressionFinalProject_files/figure-latex/appendixHistograms-6.pdf}}

\begin{verbatim}
## 
## 
## ### Plotting: diasBP ###
\end{verbatim}

\begin{verbatim}
## Warning: Removed 2 rows containing non-finite outside the scale range
## (`stat_bin()`).
\end{verbatim}

\pandocbounded{\includegraphics[keepaspectratio]{RegressionFinalProject_files/figure-latex/appendixHistograms-7.pdf}}

\begin{verbatim}
## 
## 
## ### Plotting: hr ###
\end{verbatim}

\begin{verbatim}
## Warning: Removed 1 row containing non-finite outside the scale range
## (`stat_bin()`).
\end{verbatim}

\pandocbounded{\includegraphics[keepaspectratio]{RegressionFinalProject_files/figure-latex/appendixHistograms-8.pdf}}

\begin{verbatim}
## 
## 
## ### Plotting: O2_hours_day ###
\end{verbatim}

\pandocbounded{\includegraphics[keepaspectratio]{RegressionFinalProject_files/figure-latex/appendixHistograms-9.pdf}}

\begin{verbatim}
## 
## 
## ### Plotting: bmi ###
\end{verbatim}

\pandocbounded{\includegraphics[keepaspectratio]{RegressionFinalProject_files/figure-latex/appendixHistograms-10.pdf}}

\begin{verbatim}
## 
## 
## ### Plotting: asthma ###
\end{verbatim}

\pandocbounded{\includegraphics[keepaspectratio]{RegressionFinalProject_files/figure-latex/appendixHistograms-11.pdf}}

\begin{verbatim}
## 
## 
## ### Plotting: hay_fever ###
\end{verbatim}

\pandocbounded{\includegraphics[keepaspectratio]{RegressionFinalProject_files/figure-latex/appendixHistograms-12.pdf}}

\begin{verbatim}
## 
## 
## ### Plotting: bronchitis_attack ###
\end{verbatim}

\pandocbounded{\includegraphics[keepaspectratio]{RegressionFinalProject_files/figure-latex/appendixHistograms-13.pdf}}

\begin{verbatim}
## 
## 
## ### Plotting: pneumonia ###
\end{verbatim}

\pandocbounded{\includegraphics[keepaspectratio]{RegressionFinalProject_files/figure-latex/appendixHistograms-14.pdf}}

\begin{verbatim}
## 
## 
## ### Plotting: chronic_bronchitis ###
\end{verbatim}

\pandocbounded{\includegraphics[keepaspectratio]{RegressionFinalProject_files/figure-latex/appendixHistograms-15.pdf}}

\begin{verbatim}
## 
## 
## ### Plotting: emphysema ###
\end{verbatim}

\pandocbounded{\includegraphics[keepaspectratio]{RegressionFinalProject_files/figure-latex/appendixHistograms-16.pdf}}

\begin{verbatim}
## 
## 
## ### Plotting: copd ###
\end{verbatim}

\pandocbounded{\includegraphics[keepaspectratio]{RegressionFinalProject_files/figure-latex/appendixHistograms-17.pdf}}

\begin{verbatim}
## 
## 
## ### Plotting: sleep_apnea ###
\end{verbatim}

\pandocbounded{\includegraphics[keepaspectratio]{RegressionFinalProject_files/figure-latex/appendixHistograms-18.pdf}}

\begin{verbatim}
## 
## 
## ### Plotting: SmokStartAge ###
\end{verbatim}

\pandocbounded{\includegraphics[keepaspectratio]{RegressionFinalProject_files/figure-latex/appendixHistograms-19.pdf}}

\begin{verbatim}
## 
## 
## ### Plotting: CigPerDaySmokAvg ###
\end{verbatim}

\pandocbounded{\includegraphics[keepaspectratio]{RegressionFinalProject_files/figure-latex/appendixHistograms-20.pdf}}

\begin{verbatim}
## 
## 
## ### Plotting: Duration_Smoking ###
\end{verbatim}

\begin{verbatim}
## Warning: Removed 3 rows containing non-finite outside the scale range
## (`stat_bin()`).
\end{verbatim}

\pandocbounded{\includegraphics[keepaspectratio]{RegressionFinalProject_files/figure-latex/appendixHistograms-21.pdf}}

\begin{verbatim}
## 
## 
## ### Plotting: smoking_status ###
\end{verbatim}

\pandocbounded{\includegraphics[keepaspectratio]{RegressionFinalProject_files/figure-latex/appendixHistograms-22.pdf}}

\begin{verbatim}
## 
## 
## ### Plotting: total_lung_capacity ###
\end{verbatim}

\begin{verbatim}
## Warning: Removed 1045 rows containing non-finite outside the scale range
## (`stat_bin()`).
\end{verbatim}

\pandocbounded{\includegraphics[keepaspectratio]{RegressionFinalProject_files/figure-latex/appendixHistograms-23.pdf}}

\begin{verbatim}
## 
## 
## ### Plotting: functional_residual_capacity ###
\end{verbatim}

\begin{verbatim}
## Warning: Removed 1650 rows containing non-finite outside the scale range
## (`stat_bin()`).
\end{verbatim}

\pandocbounded{\includegraphics[keepaspectratio]{RegressionFinalProject_files/figure-latex/appendixHistograms-24.pdf}}

\begin{verbatim}
## 
## 
## ### Plotting: pct_gastrapping ###
\end{verbatim}

\begin{verbatim}
## Warning: Removed 1650 rows containing non-finite outside the scale range
## (`stat_bin()`).
\end{verbatim}

\pandocbounded{\includegraphics[keepaspectratio]{RegressionFinalProject_files/figure-latex/appendixHistograms-25.pdf}}

\begin{verbatim}
## 
## 
## ### Plotting: insp_meanatt ###
\end{verbatim}

\begin{verbatim}
## Warning: Removed 1045 rows containing non-finite outside the scale range
## (`stat_bin()`).
\end{verbatim}

\pandocbounded{\includegraphics[keepaspectratio]{RegressionFinalProject_files/figure-latex/appendixHistograms-26.pdf}}

\begin{verbatim}
## 
## 
## ### Plotting: exp_meanatt ###
\end{verbatim}

\begin{verbatim}
## Warning: Removed 1650 rows containing non-finite outside the scale range
## (`stat_bin()`).
\end{verbatim}

\pandocbounded{\includegraphics[keepaspectratio]{RegressionFinalProject_files/figure-latex/appendixHistograms-27.pdf}}

\begin{verbatim}
## 
## 
## ### Plotting: FEV1_FVC_ratio ###
\end{verbatim}

\begin{verbatim}
## Warning: Removed 29 rows containing non-finite outside the scale range
## (`stat_bin()`).
\end{verbatim}

\pandocbounded{\includegraphics[keepaspectratio]{RegressionFinalProject_files/figure-latex/appendixHistograms-28.pdf}}

\begin{verbatim}
## 
## 
## ### Plotting: FEV1 ###
\end{verbatim}

\begin{verbatim}
## Warning: Removed 29 rows containing non-finite outside the scale range
## (`stat_bin()`).
\end{verbatim}

\pandocbounded{\includegraphics[keepaspectratio]{RegressionFinalProject_files/figure-latex/appendixHistograms-29.pdf}}

\begin{verbatim}
## 
## 
## ### Plotting: FVC ###
\end{verbatim}

\begin{verbatim}
## Warning: Removed 29 rows containing non-finite outside the scale range
## (`stat_bin()`).
\end{verbatim}

\pandocbounded{\includegraphics[keepaspectratio]{RegressionFinalProject_files/figure-latex/appendixHistograms-30.pdf}}

\begin{verbatim}
## 
## 
## ### Plotting: FEV1_phase2 ###
\end{verbatim}

\pandocbounded{\includegraphics[keepaspectratio]{RegressionFinalProject_files/figure-latex/appendixHistograms-31.pdf}}

\subsubsection{A. Cleaning our Data}\label{a.-cleaning-our-data}

The preprocessing steps clean and standardize the COPD dataset by
replacing all negative numeric values with missing values (NA) and
recoding several categorical variables. Unknown or missing categories in
health‐related variables (e.g., asthma, hay fever, COPD, sleep apnea)
are converted to NA, and then all Yes/No variables are transformed into
binary indicators before being stored as factors. Additional factor
conversions are applied to demographic and smoking variables. The
dataset is reorganized by moving pct\_emphysema next to the subject ID,
and a final COPD dataset is created by removing identifier and visit
information, leaving only relevant analytic variables.

\subsection{Dealing with Missing Data}\label{dealing-with-missing-data}

Before preforming any meaningful statistical analysis, it is vital that
we first analyse our missing observations and determine their nature.\\
For starters, we should note that the following variables have less then
3 missing observations: Systolic blood pressure, Diastolic blood
pressure and Heart rate. This minimal missingness is unlikely to
meaningfully impact statistical inferences or bias parameter estimates.

\subsubsection{Missing Respiratory Disease
predictors}\label{missing-respiratory-disease-predictors}

In previous analysis, it seemed like the predictors the
\textbf{explained/indicated(change)} whether said patient had that
indicated respiratory disease has missing values. However, this is not
entirely accurate as in reality those observations marked NA are
actually \textbf{unknown}. It was ultimately decided to convert those
Unknown to NA's and then impute those NA's. This `unknown' category was
not included in the model because, for most variables, it provides
little predictive information and would not meaningfully change the
results.

\subsubsection{SmokStartAge, CigPerDaySmokAvg,
Duration\_Smoking}\label{smokstartage-cigperdaysmokavg-duration_smoking}

At first glance: SmokStartAge, CigPerDaySmokAvg, Duration\_Smoking
appear to be missing as 76 observation are labeled as -1, which is
impossible based on what our predictors are explaining. However, after
further investigation it should be noted that -1 refers to not
applicable. If we observed the smoking status variable, we can see that
the missingness is explained by the fact that these 76 individuals have
never smoked, thus SmokStartAge, CigPerDaySmokAvg, Duration\_Smoking
does not apply to them. Instead of labeling these observations as NA, it
has been decided that instead they will be labeled as 0.

\subsubsection{Other missing variables}\label{other-missing-variables}

It should be noted that after further analysis, the rest of the missing
data appear to be missing \textbf{Completely at random}. Although
dataseems to be \textbf{missing completely at random}, serval variable
are missing together. pct\_emphysema, total\_lung\_capaci and
insp\_meanatt are missing together likely as a result of tests not being
conducted due to either logistical constraints (e.g., unavailability of
necessary equipment or specialized personnel) or participant-related
factors (unwillingness or inability to undergo the procedure) (likely
used the same test to calculate those three variables). The same applies
for functional\_residual\_capacity, exp\_meanatt and pct\_gastrapping as
they are always missing together. Lastly, with only 29 missing
observations FEV1\_FVC\_ratio, FEV1 and FVC are also missing together
either due to unwillingness or inability to undergo the procedure (this
applied to FEV1 and FVC only, FEV1\_FVC\_ratio is a ratio of the
previous mentioned variables and as a result is missing when the
previous variables are also missing).

\subsubsection{Missing response
variable}\label{missing-response-variable}

It should be noted that our response variable Percentage of emphysema
(damaged lung areas) denoted as pct\_emphysema has 1045 missing
observations (20\% of reponse variables are missing). Unfortunately,
dealing with missing response variable is harder then dealing with
explanatory variables as the tampering with this response variable will
likely have bigger consequences in our analysis of data. Although
opinions differ, most sources would tell us that imputing 20\% of our
response variable is dangerous, therefore our further statistical
analysis will be conducted twice: once on a dataset that imputes missing
response and once where those observations with missing response
variables will not be included.\\
\textbf{Note 1:} Data was imputed using the mice package where method =
``rf'' (Random forest imputations).\\
\textbf{Note 2:} In the original code \textbf{complete} and
\textbf{imputed} dataset where downloaded from project files. They where
imputed with the same method below, however imuting such large datasets
took a long time and it just wan't viable imputing each time we wanted
to run our R code.

\begin{Shaded}
\begin{Highlighting}[]
\CommentTok{\# {-}{-}{-}{-} Helper: build methods vector based on missingness {-}{-}{-}{-}}
\NormalTok{prepare\_methods }\OtherTok{\textless{}{-}} \ControlFlowTok{function}\NormalTok{(df, methods\_vector, }\AttributeTok{yvar =} \StringTok{"pct\_emphysema"}\NormalTok{) \{}
\NormalTok{  mvec }\OtherTok{\textless{}{-}}\NormalTok{ methods\_vector}

  \CommentTok{\# Do NOT impute Y if it has no missing data}
  \ControlFlowTok{if}\NormalTok{ (}\FunctionTok{all}\NormalTok{(}\SpecialCharTok{!}\FunctionTok{is.na}\NormalTok{(df[[yvar]]))) \{}
\NormalTok{    mvec[yvar] }\OtherTok{\textless{}{-}} \StringTok{""}
\NormalTok{  \}}

  \CommentTok{\# Do NOT impute variables with no missingness}
  \ControlFlowTok{for}\NormalTok{ (v }\ControlFlowTok{in} \FunctionTok{names}\NormalTok{(mvec)) \{}
    \ControlFlowTok{if}\NormalTok{ (}\FunctionTok{all}\NormalTok{(}\SpecialCharTok{!}\FunctionTok{is.na}\NormalTok{(df[[v]]))) \{}
\NormalTok{      mvec[v] }\OtherTok{\textless{}{-}} \StringTok{""}
\NormalTok{    \}}
\NormalTok{  \}}

  \FunctionTok{return}\NormalTok{(mvec)}
\NormalTok{\}}

\CommentTok{\# Refill cigarites with 0\textquotesingle{}s}
\NormalTok{copd }\OtherTok{=}\NormalTok{ copd }\SpecialCharTok{|\textgreater{}}
  \FunctionTok{mutate}\NormalTok{( }\AttributeTok{SmokStartAge =} \FunctionTok{ifelse}\NormalTok{( }\FunctionTok{is.na}\NormalTok{(SmokStartAge), }\DecValTok{0}\NormalTok{ , SmokStartAge),}
          \AttributeTok{CigPerDaySmokAvg =} \FunctionTok{ifelse}\NormalTok{( }\FunctionTok{is.na}\NormalTok{(CigPerDaySmokAvg), }\DecValTok{0}\NormalTok{ , CigPerDaySmokAvg),}
          \AttributeTok{Duration\_Smoking =} \FunctionTok{ifelse}\NormalTok{( }\FunctionTok{is.na}\NormalTok{(Duration\_Smoking), }\DecValTok{0}\NormalTok{ , Duration\_Smoking) )}

\CommentTok{\# Observations 468, 3351 are truely missing and not just 0}
\NormalTok{copd[ }\DecValTok{486}\NormalTok{ ,]}\SpecialCharTok{$}\StringTok{"Duration\_Smoking"} \OtherTok{=} \ConstantTok{NA}
\NormalTok{copd[ }\DecValTok{3351}\NormalTok{ ,]}\SpecialCharTok{$}\StringTok{"Duration\_Smoking"} \OtherTok{=} \ConstantTok{NA}

\CommentTok{\# Three dataframes: Complete , Imputed and non{-}missing}
\NormalTok{complete }\OtherTok{=}\NormalTok{ copd}
\NormalTok{imputed }\OtherTok{=}\NormalTok{ copd[ }\SpecialCharTok{!}\FunctionTok{is.na}\NormalTok{( copd}\SpecialCharTok{$}\NormalTok{pct\_emphysema ) , ]}
\NormalTok{non\_missing }\OtherTok{=} \FunctionTok{na.omit}\NormalTok{(copd)}

\CommentTok{\# Preprocess data data. transform pct\_emphizyme to logistic}
\NormalTok{complete }\OtherTok{=}\NormalTok{ complete }\SpecialCharTok{|\textgreater{}} 
  \FunctionTok{mutate}\NormalTok{( }\AttributeTok{pct\_emphysema =}\NormalTok{ pct\_emphysema }\SpecialCharTok{*}\NormalTok{ (}\DecValTok{1}\SpecialCharTok{/}\DecValTok{100}\NormalTok{) ) }\SpecialCharTok{|\textgreater{}}
  \FunctionTok{mutate}\NormalTok{( }\AttributeTok{pct\_emphysema =} \FunctionTok{log}\NormalTok{( (pct\_emphysema)}\SpecialCharTok{/}\NormalTok{(}\DecValTok{1}\SpecialCharTok{{-}}\NormalTok{pct\_emphysema) ) )}


\CommentTok{\# ==========================================================}
\CommentTok{\# 1. COMPLETE → Impute Y and all other missing values}
\CommentTok{\# ==========================================================}
\NormalTok{mice\_complete }\OtherTok{\textless{}{-}} \FunctionTok{mice}\NormalTok{(}
\NormalTok{  complete,}
  \AttributeTok{method =} \StringTok{"rf"}\NormalTok{,}
  \AttributeTok{m =} \DecValTok{6}\NormalTok{,}
  \AttributeTok{maxit =} \DecValTok{10}\NormalTok{,}
  \AttributeTok{seed =} \DecValTok{123}
\NormalTok{)}

\CommentTok{\# OVERWRITE original complete data frame}
\NormalTok{complete }\OtherTok{\textless{}{-}} \FunctionTok{complete}\NormalTok{(mice\_complete)}

\CommentTok{\# ==========================================================}
\CommentTok{\# 2. IMPUTED → Y already complete → only impute other vars}
\CommentTok{\# ==========================================================}
\NormalTok{methods\_imputed }\OtherTok{\textless{}{-}} \FunctionTok{prepare\_methods}\NormalTok{(imputed, methods\_vector)}

\NormalTok{mice\_imputed }\OtherTok{\textless{}{-}} \FunctionTok{mice}\NormalTok{(}
\NormalTok{  imputed,}
  \AttributeTok{method =} \StringTok{"rf"}\NormalTok{,}
  \AttributeTok{m =} \DecValTok{6}\NormalTok{,}
  \AttributeTok{maxit =} \DecValTok{10}\NormalTok{,}
  \AttributeTok{seed =} \DecValTok{123}
\NormalTok{)}

\CommentTok{\# OVERWRITE original imputed data frame}
\NormalTok{imputed }\OtherTok{\textless{}{-}} \FunctionTok{complete}\NormalTok{(mice\_imputed)}
\end{Highlighting}
\end{Shaded}

\newapge

\subsection{Boxcox justification}\label{boxcox-justification}

Although the optimal power transformation may seem arbitrary (in many
cases it is), it is a useful tool that allows us to stabilize our model
in order to conduct inference. In the plots below are a comparison
between residual plots and normality plots for transformed and
transformed data.

\begin{Shaded}
\begin{Highlighting}[]
\NormalTok{complete}
\end{Highlighting}
\end{Shaded}

\begin{verbatim}
## # A tibble: 5,747 x 33
##    pct_emphysema visit_age gender race    height_cm weight_kg sysBP diasBP    hr
##            <dbl>     <dbl> <chr>  <chr>       <dbl>     <dbl> <dbl>  <dbl> <dbl>
##  1         0.985      54.5 Female White        160.      73     130     80    87
##  2         1.70       62.3 Female White        163.      86     170     80    81
##  3         1.11       65.9 Female White        162.      62.8    96     63    66
##  4         1.56       59.6 Male   White        183.     110     142     88    75
##  5         2.05       67.5 Male   White        179.      83     106     72    72
##  6         1.98       69.8 Female White        159.      78     122     78    87
##  7         2.20       68.9 Male   Black ~      169.      51     120     60    88
##  8         1.94       60.9 Male   White        185.      83.6   124     69    71
##  9         1.51       60.9 Female White        162.      89.4   156     88    81
## 10         1.43       81   Female White        165.      74     116     78    59
## # i 5,737 more rows
## # i 24 more variables: O2_hours_day <dbl>, bmi <dbl>, asthma <fct>,
## #   hay_fever <fct>, bronchitis_attack <fct>, pneumonia <fct>,
## #   chronic_bronchitis <fct>, emphysema <fct>, copd <fct>, sleep_apnea <fct>,
## #   SmokStartAge <dbl>, CigPerDaySmokAvg <dbl>, Duration_Smoking <dbl>,
## #   smoking_status <chr>, total_lung_capacity <dbl>,
## #   functional_residual_capacity <dbl>, pct_gastrapping <dbl>, ...
\end{verbatim}

\end{document}
